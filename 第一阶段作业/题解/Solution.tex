\documentclass[12pt]{article}
\usepackage[UTF8]{ctex}
\usepackage{amsmath}
\usepackage{indentfirst}
\usepackage{graphicx}
\setlength{\parindent}{2em}
\title{IOI2020 国家集训队第一阶段作业第一部分解题报告}
\author{党星宇}
\date{2019.10}
\usepackage{fancyhdr}
\pagestyle{fancy}
\lhead{\small \leftmark}
\chead{}
\rhead{\small{解题报告}}
\lfoot{}
\cfoot{}
\rfoot{\thepage}
\renewcommand{\headrulewidth}{0.4pt}
\renewcommand{\footrulewidth}{0.4pt}

\begin{document}
\maketitle

\newpage

\section{Coloring Balls}
\subsection{试题来源}
AtCoder Regular Contest 089 F
\subsection{题目大意}
最开始有$N$个白球排成一排。有$K$次操作,每次给出红色或蓝色两种颜色中的一种,你可以选择任意一个区间将这个区间的球成这种染色。但是不能直接将白球染成蓝色。问$K$次操作完成后有多少种可能的不同球的序列。

答案对$10^9+7$取模。
\subsection{数据范围}
$N,K\le 70$
\subsection{时空限制}
时间限制:$4$秒

空间限制:$256$MB
\subsection{解题过程}
我们先考虑一个暴力做法:$3^N$枚举所有可能的颜色状态,依次判断每一种是否合法。

但是其实我们不需要枚举所有$3^N$的可能状态,因为其中有很多是等价的。

为了表示一个颜色序列,我们用$R,B,W$分别表示红色球、蓝色球和白色球。

我们以'WRRBRBBWWRRRRWRBBWWRRBBRR'为例。

首先,按照白色将原序列分成若干段:\{'RRBRBB', 'RRRR', 'RBB', 'RRBBRR'\}。

其次,我们将相同颜色的一段缩在一起:\{'RBRB', 'R', 'RB', 'RBR'\}。

然后,我们将给每一段按照'B'的个数标号:

Group 1:'R'

Group 2:'B', 'RB', 'BR', 'RBR'

Group 3:'BRB', 'RBRB', 'BRBR', 'RBRBR'

Group 4:'BRBRB', 'RBRBRB', 'BRBRBR', 'RBRBRBR'

$\cdots$

考虑除了第一次和第二次操作只能分别为$r,b$之外,其余非平凡情况中,任意一次$r$或者$b$的操作都可以增加一个$B$。具体的,每一组的操作序列如下。

Group 1:'r'

Group 2:'rb'

Group 3:'rb?'

Group 4:'rb??'

Group 5:'rb???'

$\cdots$

我们用一个数字序列$f$表示一组颜色的等价类。由于各段显然是独立的,我们可以以任意顺序排列$f$中的元素。例如例子中的$f$=['3', '2', '2', '1']。假设$g(N)$表示$N$的拆分数,总的状态数个数是$O(g(N)\times N)$的,对于$N\le 70$,至多为$418662$。

下面我们考虑如何检验一个颜色等价类是否合法,相当于是对于颜色序列中的每一段分配一个操作子序列符合他的Group的要求,显然,我们考虑贪心地分配:
\begin{itemize}
  \item 首先将$f$按从大到小排序
  \item 对于每个$k$,将$S$中的第$k$个'r'(它的位置记为$r_k$)分配给$f[k]$
  \item 对于每一个$f[k]=x$,(我们从左到右依次考虑每一个$f[i]$),如果$x\ge 2$,那么将$r_k$右边最靠左的$b$(它的位置记为$b_k$)分配给它。
  \item 对于每一个$f[k]=x$,如果$x\ge 3$,就把$b_k$右边未被使用的最近$x-2$个操作分配给它。
\end{itemize}

足以通过本题。

\newpage

\end{document} 